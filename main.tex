\documentclass[a4paper, 14pt]{extarticle}

\usepackage[brazilian]{babel}
% \usepackage[utf8]{inputenc}
\usepackage[T1]{fontenc}
\usepackage[margin=1.5cm,top=1.8cm,noheadfoot=true]{geometry}

\usepackage{float, pgf, caption, subcaption}

% \input{secoes}
% \input{teorema}
\makeatletter

\usepackage{amsthm, amsmath, amssymb, bm, mathtools}
\usepackage{enumitem, etoolbox, xpatch}
% \usepackage[mathcal]{euscript}
% \usepackage[scr]{rsfso}
\usepackage{mathptmx}
\usepackage{relsize, centernot, tikz, xcolor}


%%%% QED symbols %%%%
\def\qed@open{\ensuremath{\square}}
\def\qed@open@small{\ensuremath{\mathsmaller\qed@open}}

\def\qed@fill{\ensuremath{\blacksquare}}
\def\qed@fill@small{\ensuremath{\mathsmaller\qed@fill}}

\definecolor{qed@gray}{gray}{0.8}
\def\qed@gray{\ensuremath{\color{qed@gray}\blacksquare}}
\def\qed@gray@small{\ensuremath{\color{qed@gray}\mathsmaller\blacksquare}}

\def\qed@both@cmd#1#2{\begin{tikzpicture}[baseline=#2]
    \draw (0,0) [fill=qed@gray] rectangle (#1,#1);
\end{tikzpicture}}
\def\qed@both{\qed@both@cmd{0.6em}{0.2ex}}
\def\qed@both@small{\qed@both@cmd{1ex}{0ex}}

\def\showAllQED{
    \qed@open ~ \qed@open@small \\
    \qed@fill ~ \qed@fill@small \\
    \qed@gray ~ \qed@gray@small \\
    \qed@both ~ \qed@both@small
}

%% escoha do QED %%
\renewcommand{\qedsymbol}{\qed@fill@small}

% marcadores de prova
\newcommand{\direto}[1][~]{\ensuremath{(\rightarrow)}#1}
\newcommand{\inverso}[1][~]{\ensuremath{(\leftarrow)}#1}

% fontes
% conjunto potencia
\DeclareSymbolFont{boondox}{U}{BOONDOX-cal}{m}{n}
\DeclareMathSymbol{\pow}{\mathalpha}{boondox}{"50}

% somatorio
\DeclareSymbolFont{matext}{OMX}{cmex}{m}{n}
\DeclareMathSymbol{\sum@d}{\mathop}{matext}{"58}
\DeclareMathSymbol{\sum@t}{\mathop}{matext}{"50}
\undef\sum
\DeclareMathOperator*{\sum}{\mathchoice{\sum@d}{\sum@t}{\sum@t}{\sum@t}}
\DeclareMathOperator*{\bigsum}{\mathlarger{\mathlarger{\sum@d}}}

% phi computer modern
\DeclareMathAlphabet{\gk@mf}{OT1}{cmr}{m}{n}
\let\old@Phi\Phi
\def\Phi{\gk@mf{\old@Phi}}

% união elem por elem
\DeclareMathOperator{\wcup}{\mathaccent\cdot\cup}

% familia de conjuntos
\undef\fam
\DeclareMathAlphabet{\fam}{OMS}{cmsy}{m}{n}

% alguns símbolos
\undef\natural
\DeclareMathOperator{\real}{\mathbb{R}}
\DeclareMathOperator{\natural}{\mathbb{N}}
\DeclareMathOperator{\integer}{\mathbb{Z}}
\DeclareMathOperator{\complex}{\mathbb{C}}
\DeclareMathOperator{\rational}{\mathbb{Q}}
\def\symdif{\mathrel{\triangle}}
\def\midd{\;\middle|\;}
% \def\pow{\mathcal{P}}

% operações com mais espaçamento
\def\cupp{\mathbin{\,\cup\,}}
\def\capp{\mathbin{\,\cap\,}}

% alguns operadores
\DeclareMathOperator{\Dom}{Dom}
\DeclareMathOperator{\Img}{Im}

% marcadores de operadores
\def\inv{^{-1}}
\def\rel#1{\use@invr{\!\mathrel{#1}\!}}
\def\nrel#1{\use@invr{\!\centernot{#1}\!}}
\def\dmod#1{\ (\mathrm{mod}\ #1)}
\def\cgc#1{{\textnormal{[}#1\textnormal{]}}}
\def\cgp#1{{\textnormal{(}#1\textnormal{)}}}

% marcador com inverso reduzido
\def\use@invr#1{%
    \begingroup%
        \edef\inv{\inv\!}%
        #1%
    \endgroup%
}

% delimiters
\def\abs#1{{\lvert\,#1\,\rvert}}
% \DeclarePairedDelimiter{\abs}{\lvert}{\:\rvert}

\makeatother


% math display skip
\newcommand{\reducemathskip}[1][0.5em]{%
    \setlength{\abovedisplayskip}{1pt}%
    \setlength{\belowdisplayskip}{#1}%
    \setlength{\abovedisplayshortskip}{#1}%
    \setlength{\belowdisplayshortskip}{#1}%
}

% url linking problems
\def\url#1{\href{#1}{\texttt{#1}}}
% vermelho
\def\red#1{\textcolor{red}{#1}}

\def\lmref#1{\thmref[lema ]{#1}}

\usepackage{xparse, caption, booktabs}
\usepackage[hidelinks]{hyperref}
\usepackage[nameinlink, brazilian]{cleveref}
\crefformat{equation}{#2eq.~#1#3}
\crefformat{definition}{#2def.~#1#3}
\crefformat{proof}{#2dem.~#1#3}
\usepackage[section, newfloat]{minted}
\definecolor{sepia}{RGB}{252,246,226}
\setminted{
    bgcolor = sepia,
    % style   = pastie,
    frame   = leftline,
    autogobble,
    samepage,
    python3,
}
\setmintedinline{
    bgcolor={}
}

\theoremstyle{plain}
\newtheorem*{hypothesis}{Hipótese}
\newtheorem*{hypothesisf}{Hipótese Fortalecida}

\newtheoremstyle{definicao}% name of the style to be used
  {}% measure of space to leave above the theorem. E.g.: 3pt
  {}% measure of space to leave below the theorem. E.g.: 3pt
  {}% name of font to use in the body of the theorem
  {}% measure of space to indent
  {\bf}% name of head font
  {:}% punctuation between head and body
  {.8em}% space after theorem head; " " = normal interword space
  {\thmnote{\textbf{#3}}}% Manually specify head
\theoremstyle{definicao}
\newtheorem*{definition}{Definição}

\NewDocumentCommand{\seq}{ s m O{n} O{\in\natural} }
    {\IfBooleanTF{#1}
        {\ensuremath{\left({#2}_{#3}\right)}}
        {\ensuremath{\left({#2}_{#3}\right)_{{#3}{#4}}}}}


\usepackage{titling, titlesec, enumitem}
% \usepackage{algorithmic}
\usepackage{clrscode3e, xspace}
\title{\vspace{-2.5cm}\Large Lista de Exercícios Avaliativa 3 \\ \normalsize MC458 - 2s2020 - Tiago de Paula Alves - 187679}
\preauthor{}\author{}\postauthor{}
\predate{}\date{}\postdate{}
\posttitle{\par\end{center}\vskip-1em}

\titleformat{\section}{\large\bfseries}{\thesection}{.8em}{}
\titlespacing*{\section}{0pt}{.5em plus .2em minus .2em}{.5em plus .2em}

\newlist{casos}{enumerate}{2}
\setlist[casos]{wide,labelwidth={\parindent},listparindent={\parindent},parsep={\parskip},topsep={0pt},label=\textbf{Caso \arabic*}:}
\setlist[casos,2]{label=\textbf{Caso \arabic{casosi}\alph*}:}

\newlist{ncasos}{description}{2}
\setlist[ncasos]{wide,listparindent={\parindent},parsep={\parskip},topsep={0pt}}

\titleformat{\section}[runin]
    {\titlerule{}\vspace{1ex}\normalfont\Large\bfseries}{}{1em}{}[.]
\titleformat{\subsection}[runin]
    {\normalfont\large\bfseries}{}{1em}{}[)]

% linha final da página ou seção
\newcommand{\docline}[1][\\]{%
    #1\noindent\rule{\textwidth}{0.4pt}%
    \pagebreak%
}

\usepackage{tikz}
\usetikzlibrary{calc,trees,positioning,arrows,fit,shapes,calc}

\DeclareMathSymbol{\mlq}{\mathord}{operators}{``}
\DeclareMathSymbol{\mrq}{\mathord}{operators}{`'}

\usepackage{fancyhdr}
\pagestyle{empty}

% \usepackage{showframe}
\begin{document}

    \maketitle
    \thispagestyle{empty}

    \section{1}
    \begingroup
        % Aula 10 - 13

\begin{codebox}
    \Procname{$\proc{Mediana}(X, Y, n)$}
    \li \If $n \isequal 1$
        \Then
    \li     \Return $X[1]$
    \li \ElseIf $n \isequal 2$
        \Then
    \li     \Return $\text{máx}(X[1],~ Y[1])$
    \li \ElseNoIf
    \li     $mx \gets \left\lfloor(n + 1) / 2\right\rfloor$
    \li     $my \gets \left\lceil(n + 1) / 2\right\rceil$
    \li     \If $X[mx] < Y[my]$
            \Then
    \li         \Return $\proc{Mediana}(X[mx \twodots n], Y[1 \twodots my], my)$
    \li     \Else
    \li         \Return $\proc{Mediana}(X[1 \twodots mx], Y[my \twodots n], mx)$
            \End
        \End
\end{codebox}

\begin{proof}[Corretude]
    Suponha um inteiro positivo $n$ tal que, para todo $1 \leq k < n$ e quaisquer vetores $X$ e $Y$ ordenados e com $k$ elementos distintos, podemos encontrar um elemento mediano de $X \cup Y$. Suponha ainda dois vetores ordenados $X$ e $Y$ com $n$ elementos distintos cada, em que $X \cap Y = \varnothing$. Seja $U = X \cup Y$.

    \begin{casos}
        \item $n = 1$. Então, $c = X_1$ é elemento mediano de $U$.

        \item $n = 2$. Se $X_1 < Y_1$, como $Y_1 < Y_2$, então $c = Y_1$ deve ser um dos elementos medianos. Por outro lado, se $X_1 > Y_1$, então $Y_1 < X_1 < X_2$, ou seja, $c = X_1$ deve ser mediano. Como $X_1 = Y_1$ é impossível, temos em ambos os casos um elemento mediano $c$ de $U$.

        \item $n \geq 3$. Considere as posições intermediárias $m_1 = \lfloor (n + 1) / 2 \rfloor$ e $m_2 = \lceil (n + 1) / 2 \rceil$. Considere também os subvetores $X^{(1)} = [X_1, \ldots, X_{m_1 - 1}]$, $X^{(2)} = [X_{m_1 + 1}, \ldots, X_n]$, $Y^{(1)} = [Y_1, \ldots, Y_{m_2 - 1}]$ e $Y^{(2)} = [Y_{m_2 + 1}, \ldots, Y_n]$, de forma que $\#(X^{(1)}) = \#(Y^{(2)}) = m_1 - 1$ e $\#(X^{(2)}) = \#(Y^{(1)}) = m_2 - 1$.

        \begin{casos}
            \item $X_{m_1} < Y_{m_2}$. Note que $\{X_{m_1}\} \cup X^{(2)}$ e $Y^{(1)} \cup \{Y_{m_2}\}$ estão ordenados e têm $1 < m_2 < n$ elementos distintos. Logo, pela hipótese indutiva, temos um elemento mediano $c'$ de $U' = \{X_{m_1}\} \cup X^{(2)} \cup Y^{(1)} \cup \{Y_{m_2}\}$.

            Como $X_{m_1} \leq c'$ e $X$ está ordenado, então $X^{(1)}_i < X_{m_1} \leq c'$ para todo $1 \leq i < m_1$. Logo, $\#(U_<) = \#(U'_<) + \#(X^{(1)}) = \#(U'_<) + m_1 - 1$. Da mesma forma, $c' \leq Y_{m_2}$, então $\#(U_>) = \#(U'_>) + \#(Y^{(2)}) = \#(U'_>) + m_1 - 1$.

            Portanto, temos que $\abs{\#(U_<) - \#(U_>)} = \abs{\#(U'_<) + m_1 - 1 - \#(U'_>) - m_1 + 1} = \\ \abs{\#(U'_<) - \#(U'_>)}$. Então, $c = c'$ também é um elemento mediano de $U$.

            \item $X_{m_1} > Y_{m_2}$. Então teremos os vetores $X^{(1)} \cup \{X_{m_1}\}$ e $\{Y_{m_2}\} \cup Y^{(2)}$ ordenados e com $1 < m_1 < n$ elementos cada. Pela hipótese indutiva, temos então um elemento mediano $c'$ de  $U' = X^{(1)} \cup \{X_{m_1}\} \cup \{Y_{m_2}\} \cup Y^{(2)}$.

            Agora teremos que $Y^{(1)}_i < Y_{m_2} \leq c' \leq X_{m_1} < X^{(2)}_j$, para todos $1 \leq i, j < m_2$. Então, $\#(U_<) = \#(U'_<) + m_2 - 1$ e $\#(U_>) = \#(U'_>) + m_2 - 1$, ou seja, $\abs{\#(U_<) - \#(U_>)} = \abs{\#(U'_<) - \#(U'_>)}$.

            Portanto, $c = c'$ também é mediano de $U$.
        \end{casos}
    \end{casos}

    ~

    Por fim, podemos, em todos os casos possíveis, encontrar um elemento mediano $c$ de $U$.
\end{proof}

~

\begin{proof}[Complexidade]
    Para este algoritmo temos que o tempo de execução é:
    \begin{align*}
        T(1) &= T(2) = \Theta(1) \\
        T(n) &= \begin{cases}
            T\left(\left\lceil\frac{n + 1}{2}\right\rceil\right) + \Theta(1)
            & \text{se } X\left[\lfloor(n + 1) / 2\rfloor\right] < Y\left[\lceil(n + 1) / 2\rceil\right] \\
            T\left(\left\lfloor\frac{n + 1}{2}\right\rfloor\right) + \Theta(1)
            & \text{caso contrário}
        \end{cases}
    \end{align*}

    Em uma análise de pior caso, a relação de recorrência seria $T(n) = T\left(\left\lceil\frac{n + 1}{2}\right\rceil\right) + \Theta(1)$. Logo, pelo Teorema Master, o algoritmo tem complexidade $T(n) \in \Theta\left(n^{\log_2 1} \lg n\right) = \Theta(\lg n) \subset o(n)$.
\end{proof}

    \endgroup

    \docline[]

    \section{2}
    \begingroup
        Suponha um algoritmo que dado dois vetores ordenados $X$ e $Y$ com $n$ elementos dististintos, sendo $X \cap Y = \varnothing$, retorna a posição de algum elemento mediano de $X \cup Y$ na forma de um par $(V, i)$, onde $V$ indica qual vetor, $X$ ou $Y$, contém o elemento e $i$ é sua posição no vetor.

Note que se $X$ e $Y$ têm distribuições de valores semelhantes, então os elementos medianos estarão no entorno da posições $n/2$ de $X$ ou $Y$. Por outro lado, se $X_n < Y_1$, como eles estão ordenados, as medianas serão esses dois elementos. O mesmo vale quando $Y_n < X_1$. Portanto, dadas distribuições específicas, quaisquer posições de $X$ ou $Y$ podem ser soluções do problema. Ou seja, existem $2 n$ soluções possíveis das quais o algoritmo deve decidir entre uma das 2 corretas.

Para um modelo baseado em comparações, podemos imaginar o algoritmo como uma árvore de decisão binária, sendo as folhas as soluções do problema. Para que o algoritmo seja ótimo no pior caso, a árvore deve estar balanceada e com o menor número de folhas possíveis, mantendo $2^h$ nós à uma altura $h$ da raiz, ou seja, após $h$ comparações. Assim, teremos no útltimo nível, com $2 n$ folhas, que $2 n \leq 2^h$, ou seja, $h \geq \lg n + 1$.

Além disso, como existem duas soluções corretas, um algoritmo ótimo conseguiria decidir uma das soluções sem comparação entre elas, reduzindo a altura ótima para $h^* \geq \lg n$, mas sem alterar a complexidade.

Por fim, temos então que para todo algoritmo $A$ baseado em comparações, $T_A(n) \geq c h \geq c \lg n$ para $c > 0$ e $n$ suficientemente grande, ou seja, $T_A(n) \in \Omega(\lg n)$. Portanto, a cota inferior do problema neste modelo é $\Omega(\lg n)$.

    \endgroup

    \docline[]

    \section{3}
    \begingroup
        % Aula 15

\subsection{a} Como o algoritmo HeapSort tem complexidade $\Theta(n \lg n)$, a solução de Xitoró terá tempo de execução total de
\[
    T(n) = \sum_{i = 1}^k \Theta(n_i \lg n_i) + \Theta(1) = \Theta\left(\sum_{i = 1}^k n_i \lg n_i \right)
\]

Logo, dependendo do comportamento de $n_i$, é possível que $T(n) \in \omega(n)$, ou seja, $T(n) \not\in O(n)$. Portanto, essa solução não terá necessariamente o comportamento assintótico desejado. Os requisitos de espaço, no entanto, são alcaçados.

\subsection{b} Como o CountingSort tem uso de espaço $O(n)$, o espaço total seria $O(n^2)$, no entanto, o compartilhamento de buffer de contagem permite espaço em ordem $O(n)$. Apesar disso, a complexidade da solução de Chorãozinho terá, para este problema:
\begin{align*}
    T(n) &= \sum_{i = 1}^k O(n_i + n) + O(1) \\
    &= O\left(\sum_{i=1}^k n_i\right) + n \sum_{i=1}^k O(1) \\
    &= O(n) + n O(k) = n O(n) \\
    &= O(n^2)
\end{align*}
Portanto, também não terá o comportamento assintótico desejado.

\subsection{c} A solução desse problema, apresentada abaixo, se baseia na ideia de que um número inteiro pode ser visto como um vetor de dígitos em uma base $b$ qualquer. Assim, o algoritmo funciona como o CountingSort, contando as repetições de um número, sendo que cada vetor $V_i$ tem sua contagem em um dígito $i-1$ do contador. Para que o algoritmo seja linear,  vamos assumir um modelo computacional similar ao RAM, mas com um operação extra: $\proc{PróximoDigito}(n, b)$, que descobre o próximo dígito não-nulo de $n$ na base $b$. Além disso, a operação de potência também é assumida constante.

Isso não é muito longe de arquiteturas atuais, como x86 e ARM, com as operações \textit{find first set} ou \textit{count leading zeroes}. Uma implementação real, em uma máquina dessas, deveria aproximar $b$ para a potência de 2 mais próxima, onde $\proc{PróximoDigito}$ e a potência $b^k$ podem ser realizadas em tempo constante. No entanto, uma implementação real teria muitos limites quanto aos tamanhos $n$ e $k$ do problema se quisesse manter o comportamento assintótico do algoritmo.

\begin{codebox}
    \Procname{$\proc{CountSortVarios}(V, k, n)$}
    \li \Comment Contador de repetições de cada elemento.
    \li Seja $C[1 \twodots n] = [0 \twodots 0]$ um novo vetor. \label{linha:czero}
    \li
    \li \Comment Maior quantidade possível em um dígito.
    \li $b \gets \text{máx}\left(\attrib{V[i]}{\id{tamanho}} ~\For~ i = 1 ~\To~ k\right) + 1$ \label{linha:base}
    \li
    \li \Comment Contagem das repetições.
    \li \For $i = 1$ \To $k$ \label{linha:for}
        \Do
    \li     \For $j = 1$ \To $\attrib{V[i]}{\id{tamanho}}$ \label{linha:forcnt}
            \Do
    \li         \Comment Cada vetor $V_i$ é contado no dígito $i-1$ base $b$.
    \li         $C[V[i][j]] \gets C[V[i][j]] + 1 \cdot b^{i-1}$ \label{linha:forcnt:end}
            \End
    \li     \Comment Parte já ordenada do vetor $V_i$.
    \li     $\attrib{V[i]}{\id{tamanho}} \gets 0$ \label{linha:for:end}
        \End
    \li
    \li \Comment Reposição dos elementos nos vetores.
    \li \For $i = 1$ \To $n$ \label{linha:fornum}
        \Do
    \li     \While $C[i] > 0$ \label{linha:while}
            \Do
    \li         \Comment Próximo vetor com contagens em $C_i$.
    \li         $d \gets \proc{PróximoDigito}(C[i], b)$ \label{linha:proxd}
    \li         \Comment Quantidade de repetições de $i$ em $V_{d+1}$.
    \li         $\id{cnt} \gets C[i] / b^d ~\kw{mod}~ b$ \label{linha:cntd}
    \li         $C[i] \gets C[i] -  \id{cnt} \cdot b^d$
    \li         \For $j = 1$ \To $\id{cnt}$ \label{linha:cnt}
                \Do
    \li             $\attrib{V[d + 1]}{\id{tamanho}} \gets \attrib{V[d + 1]}{\id{tamanho}} + 1$
    \li             $V[d + 1][\attrib{V[d + 1]}{\id{tamanho}}] = i$
                \End \label{linha:cnt:end}
            \End \label{linha:while:end}
        \End \label{linha:fornum:end}
\end{codebox}

\begin{proof}[Corretude]
    Nas linhas \ref{linha:forcnt} a \ref{linha:forcnt:end} é feita a contagem das repetições dos elementos do vetor $V_i$. A contagem acontece apenas no $i$-ésimo dígito do número no vetor $C$, que é alcançado com o fator $b^{i-1}$. A invariante aqui é que $C$ contém nos dígitos $i$ todas as repetições de $V_{i,1}$ até $V_{i,j}$

    Para que seja correta a invariante, é necessário que as repetições não extrapolem o tamanho de um dígito, por isso a base foi escolhida como $\displaystyle b = 1 + \max_{1 \leq i \leq k}(n_i)$, na linha \ref{linha:base}. Além disso, a linha \ref{linha:for:end} recomeça o marcador de tamanho do vetor para 0, que agora deverá marcar a quantidade de posições já ordenadas no vetor. Com isso, temos que as linhas \ref{linha:for} a \ref{linha:for:end} garantem que $C$ tem todas as contagens do vetores $V_1, \ldots, V_i$, nos dígitos correspondentes.

    A partir disso, o laço nas linhas \ref{linha:fornum} até \ref{linha:fornum:end} é responsável por repopular os vetores de forma ordenada, mantendo a invariante de que todos os números de 1 a $i$ já foram inseridos nos vetores $V_1, \ldots, V_k$ nas ordens e quantidades corretas. Para alcaçar isso, o laço interno \ref{linha:while}-\ref{linha:while:end} percorre o contador de repetições $C_i$, recuperando o próximo dígito não nulo $d$ (\ref{linha:proxd} e \ref{linha:cntd}), que corresponde à quantidade de repetições de $i$ em $V_{d+1}$. Esse valor então é removido de $C_i$, no dígito correspondente, para manter a invariante de que $C_i$ contém apenas as repetições de $i$ ainda não inseridas em $V_1, \ldots, V_k$.

    Por fim, mantidas as invariantes, o laço \ref{linha:fornum}-\ref{linha:fornum:end} encerra com todos os números de 1 até $n$ inseridos em $V_1, \ldots, V_k$ de forma ordenada. Portanto, $V_1, \ldots, V_k$ estão ordenados.
\end{proof}

\begin{proof}[Complexidade]
    Podemos ver que a linha \ref{linha:czero} executa em $O(n)$ e a linha \ref{linha:base} executa em $O(k) = O(n)$. Os laços em \ref{linha:for}-\ref{linha:for:end} executam em:
    \[
        T_{\{L8-L13\}}(n) = \sum_{i = 1}^k n_i = O(n)
    \]

    Por fim, temos ainda os laços de \ref{linha:fornum} até \ref{linha:fornum:end}. Assumindo $\proc{PróximoDigito}$ em tempo constante, como discutido anteriormente, podemos afirmar que o laço tem mesma complexidade que o número de todas as repetições de 1 a $n$ em $V_1, \ldots, V_k$. Pela invariante de \ref{linha:for}-\ref{linha:for:end}, isso equivale à quantidade de elementos em todos os vetores. Portanto:
    \begin{align*}
        T_{\{L16-L25\}}(n) = \sum_{i = 1}^n \sum_\text{($d \in$ dígitos não-nulos de $C_i$)} \text{dígito $d$ de $C_i$} = \sum_{j = 1}^k n_j = O(n)
    \end{align*}

    Logo, a complexidade do algoritmo como um todo é:
    \[
        T(n) = O(n) + O(n) + O(n) + O(n) = O(n)
    \]

    ~

    Para os requisitos de espaço, assumindo que $C$ é capaz de armazenar qualquer inteiro em $[0, b^k)$, o algoritmo usa $O(n)$ de armazenamento na sua execução.

    Para uma implementação em uma máquina binária, o espaço para o número deveria crescer com $O(k \lg b)$, mas para pequenos valores de $k$ e $b$, isso pode ser desconsiderado.
\end{proof}

    \endgroup

    \docline[]

\end{document}
